%%%%%%%%%%%%%%%%%%%%%%%%%%%%%%%%%%%%%%%%%
% Lachaise Assignment
% LaTeX Template
% Version 1.0 (26/6/2018)
%
% This template originates from:
% http://www.LaTeXTemplates.com
%
% Authors:
% Marion Lachaise & François Févotte
% Vel (vel@LaTeXTemplates.com)
%
% License:
% CC BY-NC-SA 3.0 (http://creativecommons.org/licenses/by-nc-sa/3.0/)
% 
%%%%%%%%%%%%%%%%%%%%%%%%%%%%%%%%%%%%%%%%%

%----------------------------------------------------------------------------------------
%	PACKAGES AND OTHER DOCUMENT CONFIGURATIONS
%----------------------------------------------------------------------------------------

\documentclass{article}

\input{structure.tex} % Include the file specifying the document structure and custom commands

%----------------------------------------------------------------------------------------
%	ASSIGNMENT INFORMATION
%----------------------------------------------------------------------------------------

\title{MAE001: Modelagem Matemática em Finanças I} % Title of the assignment

\author{Ramon Duarte de Melo\\ \texttt{ramonduarte@poli.ufrj.br}} % Author name and email address

\date{Universidade Federal do Rio de Janeiro (UFRJ) --- \today} % University, school and/or department name(s) and a date

%----------------------------------------------------------------------------------------

\begin{document}

\maketitle % Print the title

%----------------------------------------------------------------------------------------
%	INTRODUCTION
%----------------------------------------------------------------------------------------

\section*{Introdução} % Unnumbered section

O objetivo do Projeto I é implementar, avaliar e comparar o algoritmo recursivo proposto pelo livro no capítulo 1.3 e o Método de Monte-Carlo, aplicados à determinação do valor de contratos americanos e europeus de opção de compra e venda, realizando comparações de cunho matemático-estatístico e produzindo gráficos com tais observações. 

Para tal, foi utilizada a linguagem \emph{Python 3.6.7}, com os módulos \emph{numpy} (métodos numéricos) e \emph{matplotlib.pyplot} (visualização de dados).
O programa requer a instalação destes módulos, mas possui uma ferramenta de instalação automatizada das dependências (\emph{pipenv}). 

O código utilizado neste trabalho, bem como o deste relatório e as imagens geradas, foi aberto e disponibilizado publicamente no repositório https://github.com/ramonduarte/mmftrab2.


%----------------------------------------------------------------------------------------
%	PROBLEM 1
%----------------------------------------------------------------------------------------

\section*{Atividade 1} % Numbered section

Nesta atividade, foi implementado o algoritmo sugerido no livro em seu capítulo 1.3.
O procedimento é composto de três etapas:

\begin{enumerate}
	\item obtenção dos valores finais, com especial atenção para evitar a explosão combinatória típica do modelo binomial.
	\item cálculo recursivo dos valores intermediários utilizando os valores finais.
	\item dedução do valor inicial $V_{0}$ do contrato, que também representa seu custo pela teoria de precificação da arbitragem.
\end{enumerate}

Os contratos escolhidos possuem os mesmos parâmetros, tanto para a opção de compra, quanto para a opção de venda:

\begin{itemize}
	\item valor inicial do ativo: 4
	\item taxa de valorização: 100%
	\item taxa de desvalorização: 50%
	\item taxa de renda fixa: 25%
	\item preço de \emph{strike}: 5
\end{itemize}

Estes parâmetros foramutilizados junto às probabilidades neutras a risco $\tilde{p} = \tilde{q} = 0.5$.
Foram calculados os valores para 10 simulações, tais que $N = 2^{k}; k = 1, 2, ..., 10$.


\begin{figure}[]
	\includegraphics[width=\linewidth*9/10]{Figure_0.png}
	\centering
	
	\label{}
\end{figure}


\begin{figure}[]
	\includegraphics[width=\linewidth*9/10]{Figure_1.png}
	\centering
	
	\label{}
\end{figure}

Opções de venda (PUT) têm sua valorização limitada pelo fato de que dependem da queda do valor do ativo.
Tipicamente, ativos costumam possuir um piso finito, abaixo do qual seu preço não pode cair (neste exemplo, zero).
Porém, o crescimento é linear no início da curva, até $k = 8$, aproximadamente. 

Opções de compra (CALL), por outro lado, possuem crescimento ilimitado.
O crescimento é exponencial mesmo utilizando uma base maior ($10$) para o valor inicial do contrato que para o tempo ($k = log_{2} \ N$ neste cenário).



%----------------------------------------------------------------------------------------
%	PROBLEM 2
%----------------------------------------------------------------------------------------


\section*{Atividade 2}

Os mesmos contratos da atividade anterior foram recalculados utilizando o Método de Montecarlo.
Desta vez, as três etapas são:

\begin{enumerate}
	\item produção dos caminhos a serem percorridos dentro da árvore de possibilidades.
	\item determinação do valor do contrato para cada caminho (repetido $M$ vezes).
	\item cálculo da média dos valores encontrados.
\end{enumerate}

Este método produz resultados estimados, ao passo que o anterior produz resultados tidos como exatos.
Por isto, para esta atividade, foi fixado o valor de $M=5000$, repetidos os contratos anteriores e comparados os resultados.
A métrica utilizada foi o erro quadrático médio, para evitar que erros simétricos se anulem.

Os contratos de venda tiveram alta variância no erro quadrático médio, indicando que, se existe, a influência da extensão do prazo (ou redução do intervalo de discretização do tempo) é pequena.

Os contratos de compra, por outro lado, tiveram crescimento exponencial similar ao do próprio valor inicial do contrato.

\begin{figure}[H]
	\includegraphics[width=\linewidth]{Figure_2.png}
	\centering
	
	\label{}
\end{figure}

\begin{figure}[H]
	\includegraphics[width=\linewidth]{Figure_3.png}
	\centering
	
	\label{}
\end{figure}

%----------------------------------------------------------------------------------------
%	PROBLEM 2
%----------------------------------------------------------------------------------------

\section*{Atividade 3}

Por fim, para esta atividade, foi escolhido o contrato europeu de compra com \emph{lookback}, descrito na página 17 do livro.
A principal diferença para um contrato europeu de compra regular é a função do valor do contrato:

\begin{equation}
	V_{n} = \max_{0 \leq j \leq n} S_{j}  -  K
\end{equation}

Os valores são idênticos ao contrato anterior, exceto pelo preço de strike $K = 0$.
A dependência do caminho está estabelecida pela necessidade de guardar o maior valor que o ativo atingiu previamente.

O \emph{lookback} provê um piso mais elevado ao contrato de opção e, por isso, o crescimento exponencial é observado novamente.
Os valores iniciais do são ainda mais elevados nesta modalidade.

No livro, os autores utilizam o algoritmo da seção 1.3 (o mesmo usado na Atividade 1), porém os tempos são significativamente mais curtos. Para $k \in [1, 2]$, os resultados foram compatíveis. 

\begin{figure}[H]
	\includegraphics[width=\linewidth]{Figure_5.png}
	\centering
	
	\label{}
\end{figure}



\end{document}
